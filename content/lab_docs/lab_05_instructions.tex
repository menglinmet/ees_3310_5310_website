% Options for packages loaded elsewhere
\PassOptionsToPackage{unicode}{hyperref}
\PassOptionsToPackage{hyphens}{url}
%
\documentclass[
]{article}
\usepackage{lmodern}
\usepackage{amssymb,amsmath}
\usepackage{ifxetex,ifluatex}
\ifnum 0\ifxetex 1\fi\ifluatex 1\fi=0 % if pdftex
  \usepackage[T1]{fontenc}
  \usepackage[utf8]{inputenc}
  \usepackage{textcomp} % provide euro and other symbols
\else % if luatex or xetex
  \usepackage{unicode-math}
  \defaultfontfeatures{Scale=MatchLowercase}
  \defaultfontfeatures[\rmfamily]{Ligatures=TeX,Scale=1}
\fi
% Use upquote if available, for straight quotes in verbatim environments
\IfFileExists{upquote.sty}{\usepackage{upquote}}{}
\IfFileExists{microtype.sty}{% use microtype if available
  \usepackage[]{microtype}
  \UseMicrotypeSet[protrusion]{basicmath} % disable protrusion for tt fonts
}{}
\makeatletter
\@ifundefined{KOMAClassName}{% if non-KOMA class
  \IfFileExists{parskip.sty}{%
    \usepackage{parskip}
  }{% else
    \setlength{\parindent}{0pt}
    \setlength{\parskip}{6pt plus 2pt minus 1pt}}
}{% if KOMA class
  \KOMAoptions{parskip=half}}
\makeatother
\usepackage{xcolor}
\IfFileExists{xurl.sty}{\usepackage{xurl}}{} % add URL line breaks if available
\IfFileExists{bookmark.sty}{\usepackage{bookmark}}{\usepackage{hyperref}}
\hypersetup{
  pdftitle={The Geochemical Carbon Cycle},
  hidelinks,
  pdfcreator={LaTeX via pandoc}}
\urlstyle{same} % disable monospaced font for URLs
\usepackage[margin=1in]{geometry}
\usepackage{graphicx,grffile}
\makeatletter
\def\maxwidth{\ifdim\Gin@nat@width>\linewidth\linewidth\else\Gin@nat@width\fi}
\def\maxheight{\ifdim\Gin@nat@height>\textheight\textheight\else\Gin@nat@height\fi}
\makeatother
% Scale images if necessary, so that they will not overflow the page
% margins by default, and it is still possible to overwrite the defaults
% using explicit options in \includegraphics[width, height, ...]{}
\setkeys{Gin}{width=\maxwidth,height=\maxheight,keepaspectratio}
% Set default figure placement to htbp
\makeatletter
\def\fps@figure{htbp}
\makeatother
\setlength{\emergencystretch}{3em} % prevent overfull lines
\providecommand{\tightlist}{%
  \setlength{\itemsep}{0pt}\setlength{\parskip}{0pt}}
\setcounter{secnumdepth}{-\maxdimen} % remove section numbering
\usepackage{mathptmx}
\usepackage[version=4]{mhchem}

\newcommand{\COO}{\ce{CO2}}
\newcommand{\methane}{\ce{CH4}}
\newcommand{\degC}{^\circ \mathrm{C}}
\newcommand{\degF}{^\circ \mathrm{F}}
\newcommand{\water}{\mathrm{H_2O}}
\newcommand{\carb}{\ce{CO3^2-}}
\newcommand{\bicarb}{\ce{HCO3-}}
\newcommand{\carbonic}{\ce{H2CO3}}
\newcommand{\Hplus}{\ce{H+}}
\newcommand{\OH}{\ce{OH-}}
\newcommand{\silica}{\ce{SiO2}}
\newcommand{\calcite}{\ce{CaCO3}}
\newcommand{\Caplus}{\ce{Ca^2+}}
\newcommand{\silicate}{\ce{SiO3^2-}}
\newcommand{\CaSi}{\ce{CaSiO3}}
\newcommand{\pH}{p\ce{H}}
\newcommand{\permil}{\permille}

\title{The Geochemical Carbon Cycle}
\author{}
\date{\vspace{-2.5em}2020-02-10}

\begin{document}
\maketitle

{
\setcounter{tocdepth}{2}
\tableofcontents
}
\hypertarget{carbon-cycle}{%
\section{Carbon Cycle}\label{carbon-cycle}}

For the following exercises, you will use the GEOCARB model, which
simulates the earth's carbon cycle.

The GEOCARB model has two time periods:

\begin{itemize}
\item
  First, it runs for 5 million years with the ``Spinup'' settings in
  order to bring the carbon cycle and climate into a steady state.
\item
  Then, at time zero, it abruptly changes the parameters to the
  ``Simulation'' settings and also dumps a ``spike'' of
  CO\textsubscript{2} into the atmosphere and runs for another 2 million
  years with the new parameters to see how the climate and carbon cycle
  adjust to the new parameters and the CO\textsubscript{2} spike.
\end{itemize}

The quantities that are graphed include:

\begin{description}
\tightlist
\item[pCO2]
is the concentration of CO\textsubscript{2} in the atmosphere, in parts
per million.
\item[WeatC]
is the rate of CO\textsubscript{2} being weathered from carbonate rocks
and moved to the
\end{description}

oceans.

\begin{description}
\tightlist
\item[BurC]
is the rate of carbonate being converted into limestone and buried on
the
\end{description}

ocean floor.

\begin{description}
\tightlist
\item[WeatS]
is the rate of SiO\textsubscript{2} being weathered from silicate rocks
and moved to the
\end{description}

oceans.

\begin{description}
\tightlist
\item[Degas]
is the rate at which CO\textsubscript{2} is released to the atmosphere
by volcanic activity
\item[tCO2]
is the total amount of CO\textsubscript{2} dissolved in the ocean,
adding all of its forms:
\end{description}

\[ \ce{\text{tco2} = [CO2] + [H2CO3] + [HCO3-] + [CO3^{2-}]}. \]

\begin{description}
\tightlist
\item[alk]
is the ocean alkalinity: the total amount of acid (\(\ce{H+}\))
necessary to
\end{description}

neutralize the carbonate and bicarbonate in the ocean. The detailed
definition is complicated, but to a good approximation,
\(\ce{\text{alk} = [HCO3-] + 2 [CO3^{2-}]}\). This is not crucial for
this lab.

\begin{description}
\tightlist
\item[CO3]
is the concentration of dissolved carbonate (\(\ce{CO3^{2-}}\)) in the
ocean,
\end{description}

in moles per cubic meter.

\begin{description}
\tightlist
\item[d13Cocn]
is the change in the fraction of the carbon-13 (\(\ce{^{13}C}\))
isotope,
\end{description}

relative to the more common carbon-12 (\(\ce{^{12}C}\)) isotope, in the
various forms of carbon dissolved in the ocean water.

\begin{description}
\tightlist
\item[d13Catm]
is the change in the fraction of \(\ce{^{13}C}\),
\end{description}

relative to \(\ce{^{12}C}\) in atmospheric CO\textsubscript{2}.

\begin{description}
\tightlist
\item[Tatm]
is the average air temperature.
\item[Tocn]
is the average temperature of ocean water.
\end{description}

\hypertarget{note}{%
\subsubsection{\texorpdfstring{\textbf{Note:}}{Note:}}\label{note}}

In this lab, you will mostly look at pCO2, but in exercise 8.2, you will
also look at the weathering.

\hypertarget{running-the-geocarb-model-from-r}{%
\subsection{Running the GEOCARB model from
R}\label{running-the-geocarb-model-from-r}}

I have provided functions for running the GEOCARB model from R:

To run the model:

\begin{verbatim}
run_geocarb(filename, co2_spike, degas_spinup, degas_sim,
plants_spinup, plants_sim, land_area_spinup, land_area_sim,
delta_t2x, million_years_ago, mean_latitude_continents)
\end{verbatim}

You need to specify \texttt{filename} (the file to save the results in)
and \texttt{co2\_spike} (the spike in CO\textsubscript{2} at time zero).

The other parameters will take default values if you don't specify them,
but you can override those defaults by giving the parameters a value.

\texttt{degas\_spinup} and \texttt{degas\_sim} are the rates of
CO\textsubscript{2} degassing from volcanoes for the spinup and
simulation phases, in trillions of molecules per year.

\texttt{plants\_spinup} and \texttt{plants\_sim} are \texttt{TRUE/FALSE}
values for whether to include the role of plants in weathering (their
roots speed up weathering by making soil more permeable and by releasing
CO\textsubscript{2} into the soil), and \texttt{land\_area} is the total
area of dry land, relative to today. The default values are:
\texttt{degas} = 7.5, \texttt{plants} = \texttt{TRUE}, and
\texttt{land\_area} = 1.

The geological configuration allows you to look into the distant past,
where the continents were in different locations and the sun was not as
bright as today.\\
\texttt{delta\_t2x} is the climate sensitivity (the amount of warming,
in degrees Celsius, that results from doubling CO\textsubscript{2}).
\texttt{million\_years\_ago} is how many million years ago you want year
zero to be and \texttt{mean\_latitude\_continents} is the mean latitude,
in degrees, of the continents (today, with most of the continents in the
Northern hemisphere, the mean latitude is 30 degrees).

After you run \texttt{run\_geocarb}, you would read the data in with
\texttt{read\_geocarb(filename)}. This function will return a data frame
with the columns \texttt{year}, \texttt{co2.total}, \texttt{co2.atmos},
\texttt{alkalinity.ocean}, \texttt{delta.13C.ocean},
\texttt{delta.13C.atmos}, \texttt{carbonate.ocean},
\texttt{carbonate.weathering}, \texttt{silicate.weathering},
\texttt{total.weathering}, \texttt{carbon.burial},
\texttt{degassing.rate}, \texttt{temp.atmos}, and \texttt{temp.ocean}.

\hypertarget{chapter-8-exercises}{%
\subsection{Chapter 8 Exercises}\label{chapter-8-exercises}}

\hypertarget{exercise-8.1-weathering-as-a-function-of-co2}{%
\subsubsection{\texorpdfstring{Exercise 8.1: Weathering as a function of
CO\textsubscript{2}}{Exercise 8.1: Weathering as a function of CO2}}\label{exercise-8.1-weathering-as-a-function-of-co2}}

In the steady state, the rate of weathering must balance the rate of
CO\textsubscript{2} degassing from the Earth, from volcanoes and
deep-sea vents.

Run a simulation with \texttt{co2\_spike} set to zero, and set the model
to increase the degassing rate at time zero (i.e., set
\texttt{degas\_sim} to a higher value than \texttt{degas\_spinup}).

\begin{enumerate}
\def\labelenumi{\alph{enumi})}
\item
  Does an increase in CO\textsubscript{2} degassing drive atmospheric
  CO\textsubscript{2} up or down? How long does it take for
  CO\textsubscript{2} to stabilize after the degassing increases at time
  zero?
\item
  How can you see that the model balances weathering against
  CO\textsubscript{2} degassing (\textbf{Hint:} what variables would you
  graph with \texttt{ggplot}?)
\item
  Repeat this run with a range of degassing values for the simulation
  phase and make a table or a graph of the equilibrium
  CO\textsubscript{2} concentration versus the degassing rate.

  Does the weathering rate always balance the degassing rate when the
  CO\textsubscript{2} concentration stabilizes?
\item
  Plot the weathering as a function of atmospheric CO\textsubscript{2}
  concentration, using the data from the model runs you did in part (c).
\end{enumerate}

\hypertarget{exercise-8.2-effect-of-solar-intensity-on-steady-state-co2-concentration}{%
\subsubsection{\texorpdfstring{Exercise 8.2: Effect of solar intensity
on steady-state CO\textsubscript{2}
concentration}{Exercise 8.2: Effect of solar intensity on steady-state CO2 concentration}}\label{exercise-8.2-effect-of-solar-intensity-on-steady-state-co2-concentration}}

The rate of weathering is a function of CO\textsubscript{2}
concentration and sunlight, and increases when either of those variables
increases. The sun used to be less intense than it is today.

Run GEOCARB with the spike set to zero, with the default values of 7.5
for both \texttt{degas\_spinup} and \texttt{degas\_sim}, and with the
clock turned back 500 million years to when the sun was cooler than
today.

What do you get for the steady state CO\textsubscript{2}? How does this
compare to what you get when you run GEOCARB for today's solar
intensity? Explain why.

\hypertarget{exercise-8.3-the-role-of-plants-graduate-students-only}{%
\subsubsection{\texorpdfstring{Exercise 8.3: The role of plants
(\textbf{Graduate students
only})}{Exercise 8.3: The role of plants (Graduate students only)}}\label{exercise-8.3-the-role-of-plants-graduate-students-only}}

The roots of plants accelerate weathering by two processes: First, as
they grow, they open up the soil, making it more permeable to air and
water. Second, the roots pump CO\textsubscript{2} down into the soil.

Run a simulation with no CO\textsubscript{2} spike at the transition and
with no plants in the spinup, but with plants present in the simulation.

\begin{enumerate}
\def\labelenumi{\alph{enumi})}
\item
  What happens to the rate of weathering when plants are introduced in
  year zero? Does it go up or down right after the transition? WHat
  happens later on?
\item
  What happens to atmospheric CO\textsubscript{2}, and why?
\item
  When the CO\textsubscript{2} concentration changes, where does the
  carbon go?
\end{enumerate}

\hypertarget{exercise-from-chapter-10}{%
\subsection{Exercise from Chapter 10}\label{exercise-from-chapter-10}}

\hypertarget{exercise-10.1-long-term-fate-of-fossil-fuel-co2}{%
\subsubsection{\texorpdfstring{Exercise 10.1: Long-term fate of fossil
fuel
CO\textsubscript{2}}{Exercise 10.1: Long-term fate of fossil fuel CO2}}\label{exercise-10.1-long-term-fate-of-fossil-fuel-co2}}

Use the GEOCARB model in its default configuration.

\begin{enumerate}
\def\labelenumi{\alph{enumi})}
\item
  Run the model with no CO\textsubscript{2} spike at the transition.
  What happens to the weathering rates (Silicate, Carbonate, and Total)
  at the transition from spinup to simulation (i.e., year zero)?
\item
  Now set the CO\textsubscript{2} spike at the transition to 1000 GTon.

  \begin{itemize}
  \item
    What happens to the weathering at the transition? How does
    weathering change over time after the transition?
  \item
    How long does it take for CO\textsubscript{2} to roughly stabilize
    (stop changing)?
  \end{itemize}
\item
  In the experiment from (b), how do the rates of total weathering and
  carbonate burial change over time?

  \begin{itemize}
  \item
    Plot what happens from shortly before the transition until 10,000
    years afterward (\textbf{Hint:} you may want to add the following to
    your \texttt{ggplot} command: \texttt{xlim(NA,1E4)} to limit the
    range of the \emph{x}-axis, or
    \texttt{scale\_x\_continuous(limits\ =\ c(NA,1E4),\ labels\ =\ comma))}
    if you also want to format the numbers on the \emph{x}-axis with
    commas to indicate thousands and millions.)

    How do the two rates change? What do you think is happening to cause
    this?
  \item
    Now plot the carbon burial and total weathering for the range 1
    million years to 2 million years. How do the two rates compare?
  \end{itemize}
\end{enumerate}

\end{document}
